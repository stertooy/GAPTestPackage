% generated by GAPDoc2LaTeX from XML source (Frank Luebeck)
\documentclass[a4paper,11pt]{report}

\usepackage[top=37mm,bottom=37mm,left=27mm,right=27mm]{geometry}
\sloppy
\pagestyle{myheadings}
\usepackage{amssymb}
\usepackage[utf8]{inputenc}
\usepackage{makeidx}
\makeindex
\usepackage{color}
\definecolor{FireBrick}{rgb}{0.5812,0.0074,0.0083}
\definecolor{RoyalBlue}{rgb}{0.0236,0.0894,0.6179}
\definecolor{RoyalGreen}{rgb}{0.0236,0.6179,0.0894}
\definecolor{RoyalRed}{rgb}{0.6179,0.0236,0.0894}
\definecolor{LightBlue}{rgb}{0.8544,0.9511,1.0000}
\definecolor{Black}{rgb}{0.0,0.0,0.0}

\definecolor{linkColor}{rgb}{0.0,0.0,0.554}
\definecolor{citeColor}{rgb}{0.0,0.0,0.554}
\definecolor{fileColor}{rgb}{0.0,0.0,0.554}
\definecolor{urlColor}{rgb}{0.0,0.0,0.554}
\definecolor{promptColor}{rgb}{0.0,0.0,0.589}
\definecolor{brkpromptColor}{rgb}{0.589,0.0,0.0}
\definecolor{gapinputColor}{rgb}{0.589,0.0,0.0}
\definecolor{gapoutputColor}{rgb}{0.0,0.0,0.0}

%%  for a long time these were red and blue by default,
%%  now black, but keep variables to overwrite
\definecolor{FuncColor}{rgb}{0.0,0.0,0.0}
%% strange name because of pdflatex bug:
\definecolor{Chapter }{rgb}{0.0,0.0,0.0}
\definecolor{DarkOlive}{rgb}{0.1047,0.2412,0.0064}


\usepackage{fancyvrb}

\usepackage{mathptmx,helvet}
\usepackage[T1]{fontenc}
\usepackage{textcomp}


\usepackage[
            pdftex=true,
            bookmarks=true,        
            a4paper=true,
            pdftitle={Written with GAPDoc},
            pdfcreator={LaTeX with hyperref package / GAPDoc},
            colorlinks=true,
            backref=page,
            breaklinks=true,
            linkcolor=linkColor,
            citecolor=citeColor,
            filecolor=fileColor,
            urlcolor=urlColor,
            pdfpagemode={UseNone}, 
           ]{hyperref}

\newcommand{\maintitlesize}{\fontsize{50}{55}\selectfont}

% write page numbers to a .pnr log file for online help
\newwrite\pagenrlog
\immediate\openout\pagenrlog =\jobname.pnr
\immediate\write\pagenrlog{PAGENRS := [}
\newcommand{\logpage}[1]{\protect\write\pagenrlog{#1, \thepage,}}
%% were never documented, give conflicts with some additional packages

\newcommand{\GAP}{\textsf{GAP}}

%% nicer description environments, allows long labels
\usepackage{enumitem}
\setdescription{style=nextline}

%% depth of toc
\setcounter{tocdepth}{1}

\usepackage{amsmath}



%% command for ColorPrompt style examples
\newcommand{\gapprompt}[1]{\color{promptColor}{\bfseries #1}}
\newcommand{\gapbrkprompt}[1]{\color{brkpromptColor}{\bfseries #1}}
\newcommand{\gapinput}[1]{\color{gapinputColor}{#1}}


\begin{document}

\logpage{[ 0, 0, 0 ]}
\begin{titlepage}
\mbox{}\vfill

\begin{center}{\maintitlesize \textbf{ SmallClassNr \mbox{}}}\\
\vfill

\hypersetup{pdftitle= SmallClassNr }
\markright{\scriptsize \mbox{}\hfill  SmallClassNr  \hfill\mbox{}}
{\Huge \textbf{ Library of groups with small class number \mbox{}}}\\
\vfill

{\Huge  1.3.1 \mbox{}}\\[1cm]
{ 9 September 2025 \mbox{}}\\[1cm]
\mbox{}\\[2cm]
{\Large \textbf{ Sam Tertooy\\
    \mbox{}}}\\
\hypersetup{pdfauthor= Sam Tertooy\\
    }
\end{center}\vfill

\mbox{}\\
{\mbox{}\\
\small \noindent \textbf{ Sam Tertooy\\
    }  Email: \href{mailto://sam.tertooy@kuleuven.be} {\texttt{sam.tertooy@kuleuven.be}}\\
  Homepage: \href{https://stertooy.github.io/} {\texttt{https://stertooy.github.io/}}\\
  Address: \begin{minipage}[t]{8cm}\noindent
 Wiskunde\\
 KU Leuven, Kulak Kortrijk Campus\\
 Etienne Sabbelaan 53\\
 8500 Kortrijk\\
 Belgium\\
 \\
 \end{minipage}
}\\
\end{titlepage}

\newpage\setcounter{page}{2}
{\small 
\section*{Abstract}
\logpage{[ 0, 0, 1 ]}
 The \textsc{SmallClassNr} package provides access to finite groups with small class number. Currently,
the package contains the finite groups of class number at most 14. \mbox{}}\\[1cm]
{\small 
\section*{Copyright}
\logpage{[ 0, 0, 2 ]}
 {\copyright} 2022\texttt{\symbol{45}}2025 Sam Tertooy 

 The \textsc{SmallClassNr} package is free software, it may be redistributed and/or modified under the
terms and conditions of the \href{ https://www.gnu.org/licenses/old-licenses/gpl-2.0.en.html} {GNU Public License Version 2} or (at your option) any later version. \mbox{}}\\[1cm]
{\small 
\section*{Acknowledgements}
\logpage{[ 0, 0, 3 ]}
 This documentation was created using the \textsc{GAPDoc} and \textsc{AutoDoc} packages. \mbox{}}\\[1cm]
\newpage

\def\contentsname{Contents\logpage{[ 0, 0, 4 ]}}

\tableofcontents
\newpage

     
\chapter{\textcolor{Chapter }{The SmallClassNr package}}\label{Chapter_The_packagename_package}
\logpage{[ 1, 0, 0 ]}
\hyperdef{L}{X7EC4E71A8078695B}{}
{
  This is the manual for the \textsf{GAP} 4 package \textsf{SmallClassNr} version 1.3.1, developed by Sam Tertooy. 
\section{\textcolor{Chapter }{Installation}}\label{Chapter_The_packagename_package_Section_Installation}
\logpage{[ 1, 1, 0 ]}
\hyperdef{L}{X8360C04082558A12}{}
{
  You can download \textsf{SmallClassNr} as a .tar.gz archive \href{https://github.com/stertooy/SmallClassNr/releases/download/v1.3.1/SmallClassNr-1.3.1.tar.gz} {here}. After extracting, you should place it in a suitable \texttt{pkg} folder. For example, on a Debian\texttt{\symbol{45}}based Linux distribution
(e.g. Ubuntu, Mint), you can place it in \texttt{\$HOME/.gap/pkg} (recommended) which makes it available for just yourself, or in the \textsf{GAP} installation directory (\texttt{gap\texttt{\symbol{45}}X.Y.Z/pkg}) which makes it available for all users. 

 You can use the following command to efficiently install the package for
yourself: 
\begin{Verbatim}[commandchars=!@A,fontsize=\small,frame=single,label=Command]
  wget -qO- https://[...].tar.gz | tar xzf - --one-top-level=$HOME/.gap/pkg
\end{Verbatim}
 If the \textsf{PackageManager} package is installed and loaded, you can install \textsf{SmallClassNr} from within a \textsf{GAP} session using \texttt{InstallPackage} (\textbf{PackageManager: InstallPackage}). 
\begin{Verbatim}[commandchars=!@|,fontsize=\small,frame=single,label=Example]
  !gapprompt@gap>| !gapinput@InstallPackage( "https://[...].tar.gz" );|
  ...
  true
\end{Verbatim}
 }

 
\section{\textcolor{Chapter }{Loading}}\label{Chapter_The_packagename_package_Section_Loading}
\logpage{[ 1, 2, 0 ]}
\hyperdef{L}{X861ED1338181C66D}{}
{
  Once installed, loading \textsf{SmallClassNr} can be done by using \texttt{LoadPackage} (\textbf{Reference: LoadPackage}). 
\begin{Verbatim}[commandchars=!@|,fontsize=\small,frame=single,label=Example]
  !gapprompt@gap>| !gapinput@LoadPackage( "SmallClassNr" );|
  ...
  true
\end{Verbatim}
 }

 
\section{\textcolor{Chapter }{Citing}}\label{Chapter_The_packagename_package_Section_Citing}
\logpage{[ 1, 3, 0 ]}
\hyperdef{L}{X7A178B0587668C3E}{}
{
  If you use the \textsf{SmallClassNr} package in your research, we would love to hear about your work via an email
to the address \href{mailto://sam.tertooy@kuleuven.be} {\texttt{sam.tertooy@kuleuven.be}}. If you have used the \textsf{SmallClassNr} package in the preparation of a paper and wish to refer to it, please cite it
as described below. 

 In Bib{\TeX}: 
\begin{Verbatim}[commandchars=!|B,fontsize=\small,frame=single,label=BibTeX]
  @misc{SCN1.3.1,
      author =       {Tertooy, Sam},
      title =        {{SmallClassNr,
                      Library of groups with small class number,
                      Version 1.3.1}},
      note =         {GAP package},
      year =         {2025},
      howpublished = {\url{https://stertooy.github.io/SmallClassNr}}
  }
\end{Verbatim}


 In Bib{\LaTeX}: 
\begin{Verbatim}[commandchars=!|B,fontsize=\small,frame=single,label=BibLaTeX]
  @software{SCN1.3.1,
      author =   {Tertooy, Sam},
      title =    {SmallClassNr},
      subtitle = {Library of groups with small class number},
      version =  {1.3.1},
      note =     {GAP package},
      year =     {2025},
      url =      {https://stertooy.github.io/SmallClassNr}
  }
\end{Verbatim}
 }

 
\section{\textcolor{Chapter }{Support}}\label{Chapter_The_packagename_package_Section_Support}
\logpage{[ 1, 4, 0 ]}
\hyperdef{L}{X7B689C0284AC4296}{}
{
  If you encounter any problems, please submit them to the \href{https://github.com/stertooy/SmallClassNr/issues} {issue tracker}. If you have any questions on the usage or functionality of \textsf{SmallClassNr}, you may contact me via email at \href{mailto://sam.tertooy@kuleuven.be} {\texttt{sam.tertooy@kuleuven.be}}. }

 }

   
\chapter{\textcolor{Chapter }{Mathematical Background}}\label{Chapter_Mathematical_Background}
\logpage{[ 2, 0, 0 ]}
\hyperdef{L}{X7EF1B6708069B0C7}{}
{
  The \emph{class number} $k(G)$ of a group $G$ is the number of conjugacy classes of $G$. In 1903, Landau proved in \cite{land03-a} that for every $n \in \mathbb{N}$, there are only finitely many finite groups with exactly $n$ conjugacy classes. The \textsf{SmallClassNr} package provides access to the finite groups with class number at most $14$. 

 These groups were classified in the following papers: 
\begin{itemize}
\item  $k(G) \leq 5$, by Miller in \cite{mill11-a} and independently by Burnside in \cite{burn11-a}
\item  $k(G) = 6,7$, by Poland in \cite{pola68-a}
\item  $k(G) = 8$, by Kosvintsev in \cite{kosv74-a}
\item  $k(G) = 9$, by Odincov and Starostin in \cite{os76-a}
\item  $k(G) = 10,11$, by Vera L{\a'o}pez and Vera L{\a'o}pez in \cite{ll85-a} (1) 
\item  $k(G) = 12$, by Vera L{\a'o}pez and Vera L{\a'o}pez in \cite{ll86-a} (2) 
\item  $k(G) = 13, 14$, by Vera L{\a'o}pez and Sangroniz in \cite{vs07-a}
\end{itemize}
 

 

 (1) In \cite{ll85-a}, three distinct groups of the form $(C_5 \times C_5) \rtimes C_4$ order $100$ with class number $10$ are given. However, only two such groups exist, being the ones with \texttt{IdClassNr} equal to \texttt{[10,25]} and \texttt{[10,26]}. 

 (2) In \cite{ll86-a}, only 48 groups with class number 12 are listed. The three missing groups are
provided in the appendix of \cite{vs07-a}. These are the groups with \texttt{IdClassNr} equal to \texttt{[12,13]}, \texttt{[12,16]} and \texttt{[12,39]}. }

   
\chapter{\textcolor{Chapter }{The Small Class Number Library}}\label{Chapter_The_Small_Class_Number_Library}
\logpage{[ 3, 0, 0 ]}
\hyperdef{L}{X7D2B74F9802686C1}{}
{
  
\section{\textcolor{Chapter }{Functions}}\label{Chapter_The_Small_Class_Number_Library_Section_Functions}
\logpage{[ 3, 1, 0 ]}
\hyperdef{L}{X86FA580F8055B274}{}
{
  

\subsection{\textcolor{Chapter }{SmallClassNrGroup}}
\logpage{[ 3, 1, 1 ]}\nobreak
\hyperdef{L}{X7F0BD3117CA1B4FE}{}
{\noindent\textcolor{FuncColor}{$\triangleright$\enspace\texttt{SmallClassNrGroup({\mdseries\slshape k, i})\index{SmallClassNrGroup@\texttt{SmallClassNrGroup}}
\label{SmallClassNrGroup}
}\hfill{\scriptsize (function)}}\\
\textbf{\indent Returns:\ }
the \mbox{\texttt{\mdseries\slshape i}}\texttt{\symbol{45}}th finite group of class number \mbox{\texttt{\mdseries\slshape k}} in the library. 



 Alternatively, the pair \texttt{[ \mbox{\texttt{\mdseries\slshape k}}, \mbox{\texttt{\mdseries\slshape i}} ]} can be given as a single argument \mbox{\texttt{\mdseries\slshape id}}. If the group is solvable, it is given as a PcGroup whose Pcgs is a
SpecialPcgs. If the group is not solvable, it will be given as a permutation
group of minimal permutation degree and with a minimal generating set. }

 
\begin{Verbatim}[commandchars=!@|,fontsize=\small,frame=single,label=Example]
  !gapprompt@gap>| !gapinput@G := SmallClassNrGroup( 6, 4 );|
  <pc group of size 18 with 3 generators>
  !gapprompt@gap>| !gapinput@NrConjugacyClasses( G );|
  6
  !gapprompt@gap>| !gapinput@IsDihedralGroup( G );|
  true
\end{Verbatim}
 

\subsection{\textcolor{Chapter }{SmallClassNrGroupsAvailable}}
\logpage{[ 3, 1, 2 ]}\nobreak
\hyperdef{L}{X79C6FA9D850D3AB1}{}
{\noindent\textcolor{FuncColor}{$\triangleright$\enspace\texttt{SmallClassNrGroupsAvailable({\mdseries\slshape k})\index{SmallClassNrGroupsAvailable@\texttt{SmallClassNrGroupsAvailable}}
\label{SmallClassNrGroupsAvailable}
}\hfill{\scriptsize (function)}}\\
\textbf{\indent Returns:\ }
\texttt{true} if the finite groups of class number \mbox{\texttt{\mdseries\slshape k}} are available in the library, and \texttt{false} otherwise. 



 

 }

 
\begin{Verbatim}[commandchars=!@|,fontsize=\small,frame=single,label=Example]
  !gapprompt@gap>| !gapinput@SmallClassNrGroupsAvailable( 14 );|
  true
  !gapprompt@gap>| !gapinput@SmallClassNrGroupsAvailable( 15 );|
  false
\end{Verbatim}
 

\subsection{\textcolor{Chapter }{AllSmallClassNrGroups}}
\logpage{[ 3, 1, 3 ]}\nobreak
\hyperdef{L}{X87052993834D10C5}{}
{\noindent\textcolor{FuncColor}{$\triangleright$\enspace\texttt{AllSmallClassNrGroups({\mdseries\slshape arg})\index{AllSmallClassNrGroups@\texttt{AllSmallClassNrGroups}}
\label{AllSmallClassNrGroups}
}\hfill{\scriptsize (function)}}\\
\textbf{\indent Returns:\ }
all finite groups with certain properties as specified by \mbox{\texttt{\mdseries\slshape arg}}. 



 The arguments must come in pairs consisting of a function and a value (or list
of possible values). At least one of the functions must be \texttt{NrConjugacyClasses}. Missing functions will be interpreted as \texttt{NrConjugacyClasses}, missing values as \texttt{true}. }

 
\begin{Verbatim}[commandchars=!@|,fontsize=\small,frame=single,label=Example]
  !gapprompt@gap>| !gapinput@L1 := AllSmallClassNrGroups( [3..5], IsNilpotent );|
  [ <pc group of size 3 with 1 generator>,
    <pc group of size 4 with 2 generators>,
    <pc group of size 4 with 2 generators>,
    <pc group of size 5 with 1 generator>,
    <pc group of size 8 with 3 generators>,
    <pc group of size 8 with 3 generators> ]
  !gapprompt@gap>| !gapinput@List( L1, NrConjugacyClasses );|
  [ 3, 4, 4, 5, 5, 5 ]
  !gapprompt@gap>| !gapinput@L2 := AllSmallClassNrGroups( IsSolvable, true, NrConjugacyClasses, 6 );|
  [ <pc group of size 6 with 2 generators>,
    <pc group of size 12 with 3 generators>,
    <pc group of size 12 with 3 generators>,
    <pc group of size 18 with 3 generators>,
    <pc group of size 18 with 3 generators>,
    <pc group of size 36 with 4 generators>,
    <pc group of size 72 with 5 generators> ]
  !gapprompt@gap>| !gapinput@ForAll( L2, G -> IsSolvable( G ) and NrConjugacyClasses( G ) = 6 );|
  true
\end{Verbatim}
 

\subsection{\textcolor{Chapter }{OneSmallClassNrGroup}}
\logpage{[ 3, 1, 4 ]}\nobreak
\hyperdef{L}{X7AD22E3083C09796}{}
{\noindent\textcolor{FuncColor}{$\triangleright$\enspace\texttt{OneSmallClassNrGroup({\mdseries\slshape arg})\index{OneSmallClassNrGroup@\texttt{OneSmallClassNrGroup}}
\label{OneSmallClassNrGroup}
}\hfill{\scriptsize (function)}}\\
\textbf{\indent Returns:\ }
one finite group with certain properties as specified by \mbox{\texttt{\mdseries\slshape arg}}. 



 The arguments must come in pairs consisting of a function and a value (or list
of possible values). At least one of the functions must be \texttt{NrConjugacyClasses}. Missing functions will be interpreted as \texttt{NrConjugacyClasses}, missing values as \texttt{true}. }

 
\begin{Verbatim}[commandchars=!@|,fontsize=\small,frame=single,label=Example]
  !gapprompt@gap>| !gapinput@H := OneSmallClassNrGroup( 6, IsAbelian );|
  <pc group of size 6 with 2 generators>
  !gapprompt@gap>| !gapinput@IsCyclic( H );|
  true
  !gapprompt@gap>| !gapinput@K := OneSmallClassNrGroup( 10, IsSolvable, true, IsNilpotent, false );|
  <pc group of size 28 with 3 generators>
  !gapprompt@gap>| !gapinput@NrConjugacyClasses( K ) = 10 and IsSolvable( K ) and not IsNilpotent( K );|
  true
\end{Verbatim}
 

\subsection{\textcolor{Chapter }{NrSmallClassNrGroups}}
\logpage{[ 3, 1, 5 ]}\nobreak
\hyperdef{L}{X84004D087859AF4F}{}
{\noindent\textcolor{FuncColor}{$\triangleright$\enspace\texttt{NrSmallClassNrGroups({\mdseries\slshape arg})\index{NrSmallClassNrGroups@\texttt{NrSmallClassNrGroups}}
\label{NrSmallClassNrGroups}
}\hfill{\scriptsize (function)}}\\
\textbf{\indent Returns:\ }
the number of finite groups with certain properties as specified by \mbox{\texttt{\mdseries\slshape arg}}. 



 The arguments must come in pairs consisting of a function and a value (or list
of possible values). At least one of the functions must be \texttt{NrConjugacyClasses}. Missing functions will be interpreted as \texttt{NrConjugacyClasses}, missing values as \texttt{true}. }

 
\begin{Verbatim}[commandchars=!@|,fontsize=\small,frame=single,label=Example]
  !gapprompt@gap>| !gapinput@NrSmallClassNrGroups( 14 );|
  93
  !gapprompt@gap>| !gapinput@NrSmallClassNrGroups( [3..5], IsNilpotentGroup );|
  6
  !gapprompt@gap>| !gapinput@NrSmallClassNrGroups( IsSolvable, true, NrConjugacyClasses, 6 );|
  7
\end{Verbatim}
 

\subsection{\textcolor{Chapter }{IteratorSmallClassNrGroups}}
\logpage{[ 3, 1, 6 ]}\nobreak
\hyperdef{L}{X7D2BB4B379563D9A}{}
{\noindent\textcolor{FuncColor}{$\triangleright$\enspace\texttt{IteratorSmallClassNrGroups({\mdseries\slshape arg})\index{IteratorSmallClassNrGroups@\texttt{IteratorSmallClassNrGroups}}
\label{IteratorSmallClassNrGroups}
}\hfill{\scriptsize (function)}}\\
\textbf{\indent Returns:\ }
an iterator that iterates over the finite groups with properties as specified
by \mbox{\texttt{\mdseries\slshape arg}}. The arguments must come in pairs consisting of a function and a value (or
list of possible values). At least one of the functions must be \texttt{NrConjugacyClasses}. Missing functions will be interpreted as \texttt{NrConjugacyClasses}, missing values as \texttt{true}. 



 

 }

 
\begin{Verbatim}[commandchars=!@|,fontsize=\small,frame=single,label=Example]
  !gapprompt@gap>| !gapinput@iter := IteratorSmallClassNrGroups( IsSolvable, false, 11 );|
  <iterator>
  !gapprompt@gap>| !gapinput@for G in iter do Print( Size( G ), "\n" ); od;|
  336
  720
  720
  1344
  1344
  1512
  2448
  29120
\end{Verbatim}
 

\subsection{\textcolor{Chapter }{IdClassNr}}
\logpage{[ 3, 1, 7 ]}\nobreak
\hyperdef{L}{X8651368082BCE413}{}
{\noindent\textcolor{FuncColor}{$\triangleright$\enspace\texttt{IdClassNr({\mdseries\slshape G})\index{IdClassNr@\texttt{IdClassNr}}
\label{IdClassNr}
}\hfill{\scriptsize (attribute)}}\\
\textbf{\indent Returns:\ }
the \textsf{SmallClassNr} ID of \mbox{\texttt{\mdseries\slshape G}}, i.e. a pair \texttt{[ \mbox{\texttt{\mdseries\slshape k}}, \mbox{\texttt{\mdseries\slshape i}} ]} such that \mbox{\texttt{\mdseries\slshape G}} is isomorphic to \texttt{SmallClassNrGroup( \mbox{\texttt{\mdseries\slshape k}}, \mbox{\texttt{\mdseries\slshape i}} )}. 



 

 }

 
\begin{Verbatim}[commandchars=!@|,fontsize=\small,frame=single,label=Example]
  !gapprompt@gap>| !gapinput@IdClassNr( AlternatingGroup( 5 ) );|
  [ 5, 8 ]
  !gapprompt@gap>| !gapinput@A := SmallClassNrGroup( 5, 8 );|
  Group([ (1,2,3), (1,4,5) ])
  !gapprompt@gap>| !gapinput@IsAlternatingGroup( A );|
  true
\end{Verbatim}
 }

 }

 \def\bibname{References\logpage{[ "Bib", 0, 0 ]}
\hyperdef{L}{X7A6F98FD85F02BFE}{}
}

\bibliographystyle{alpha}
\bibliography{manual.bib}

\addcontentsline{toc}{chapter}{References}

\def\indexname{Index\logpage{[ "Ind", 0, 0 ]}
\hyperdef{L}{X83A0356F839C696F}{}
}

\cleardoublepage
\phantomsection
\addcontentsline{toc}{chapter}{Index}


\printindex

\immediate\write\pagenrlog{["Ind", 0, 0], \arabic{page},}
\newpage
\immediate\write\pagenrlog{["End"], \arabic{page}];}
\immediate\closeout\pagenrlog
\end{document}
